%%%%%%%%%%%%%%%%%%%%%%%%%%%%%%%%%%%%%%%%%%%%%%%%%%%%%%%%%%%%%%%%%%%%%%%%
% Plantilla TFG/TFM
% Escuela Politécnica Superior de la Universidad de Alicante
% Realizado por: Jose Manuel Requena Plens
% Contacto: info@jmrplens.com / Telegram:@jmrplens
%%%%%%%%%%%%%%%%%%%%%%%%%%%%%%%%%%%%%%%%%%%%%%%%%%%%%%%%%%%%%%%%%%%%%%%%

\chapter*{Preámbulo}
\thispagestyle{empty}
\par La pobreza en España es un problema social que ha afectado a miles de familias a lo largo de los años, muy duramente por la crisis de 2008 y, actualmente, por la pandemia de COVID-19.
En esta situación y a pesar de las ayudas económicas del gobierno, muchas personas se verán dramáticamente perjudicadas, afectando incluso a la adquisición de alimentos básicos.
Los economatos sociales son proyectos que suelen ser llevados en comunidad, con pocos recursos económicos, y que permiten el acceso a recursos básicos por un precio inferior al de mercado. Sin embargo, estos proyectos tienen un volumen considerable de usuarios que se magnifica en épocas de crisis como la que ha ocasionado la pandemia.
\vspace{1em}
\par El desarrollo del proyecto se ha llevado a cabo a través de un estrecho feedback directamente con la gente que trabajará con este software; habiendo creado desde cero una solución adhoc que busca encajar lo mejor posible con las necesidades que tienen los voluntarios que atienden a las familias que acuden al economato.
\vspace{1em}
\par Se ha pretendido en todo momento mantener una infraestructura basada en la alta disponibilidad, automatización de procesos manuales y de bajo coste económico.
\vspace{1em}
\par Se ha buscado la máxima simplicidad de la experiencia de usuario, haciendo uso de diseños minimalistas e interacciones cortas, debido a la posibilidad de que haya voluntarios con poca educación digital y la necesidad que éstos puedan ser capaces de usar la aplicación o aprender a usarla con facilidad.

\cleardoublepage %salta a nueva página impar
\chapter*{Agradecimientos\footnote{Por si alguien siente curiosidad, lo que me une a Vanesa es algo más que un anillo, es un vínculo y una complicidad que aún no han conocido límites.}
}

\thispagestyle{empty}
\vspace{1cm}

\par Este trabajo no habría sido posible sin el apoyo y el estímulo de mi compañera y amiga, Vanesa, cuya amabilidad y dedicación me ha permitido mantener la concentración y motivación que un proyecto como este necesita para llegar a buen puerto.
\vspace{1em}
\par No puedo dejar de agradecer a mis tutores John y a José la predisposición y disponibilidad mostradas desde el primer día que les contacté para empezar a trabajar en este TFG. Una flexibilidad que me han permitido exprimir mis conocimientos y habilidades para conseguir construir algo de lo que sentirme orgulloso; y que, con suerte, permitirá facilitarle la labor lo suficiente a los economatos sociales para que su preocupación sea ayudar a las personas y no qué tecnología usar para ello.
\vspace{1em}
\par Es a estas personas a quien dedico este trabajo.

\cleardoublepage %salta a nueva página impar
% Aquí va la dedicatoria si la hubiese. Si no, comentar la(s) linea(s) siguientes
%\chapter*{}
%\setlength{\leftmargin}{0.5\textwidth}
%\setlength{\parsep}{0cm}
%\addtolength{\topsep}{0.5cm}
%\begin{flushright}
%\small\em{
%A mi esposa Marganit, y a mis hijos Ella Rose y Daniel Adams,\\
%sin los cuales habría podido acabar este libro dos años antes \footnote{Dedicatoria de Joseph J. Roman en %"An Introduction to Algebraic Topology"}
%}
%\end{flushright}


%\cleardoublepage %salta a nueva página impar
% Aquí va la cita célebre si la hubiese. Si no, comentar la(s) linea(s) siguientes
%\chapter*{}
%\setlength{\leftmargin}{0.5\textwidth}
%\setlength{\parsep}{0cm}
%\addtolength{\topsep}{0.5cm}
%\begin{flushright}
%\small\em{
%Si consigo ver más lejos\\
%es porque he conseguido auparme\\ 
%a hombros de gigantes
%}
%\end{flushright}
%\begin{flushright}
%\small{
%Isaac Newton.
%}
%\end{flushright}
%\cleardoublepage %salta a nueva página impar
