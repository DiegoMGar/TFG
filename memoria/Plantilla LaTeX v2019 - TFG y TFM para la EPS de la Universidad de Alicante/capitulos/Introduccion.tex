%%%%%%%%%%%%%%%%%%%%%%%%%%%%%%%%%%%%%%%%%%%%%%%%%%%%%%%%%%%%%%%%%%%%%%%%
% Plantilla TFG/TFM
% Escuela Politécnica Superior de la Universidad de Alicante
% Realizado por: Jose Manuel Requena Plens
% Contacto: info@jmrplens.com / Telegram:@jmrplens
%%%%%%%%%%%%%%%%%%%%%%%%%%%%%%%%%%%%%%%%%%%%%%%%%%%%%%%%%%%%%%%%%%%%%%%%

\chapter{Introducción}
En este apartado se va a presentar el trabajo, se introducirá la situación social actual que ha hecho de este trabajo una necesidad, qué lo ha motivado y, por último, se relacionará el alcance del trabajo y las asignaturas cursadas en el grado de Ingeniería Informática de la Universidad de Alicante.
%\vspace{1em}
\par El objetivo de este proyecto consiste en aportar una solución informática a los economatos sociales, dado que usan procedimientos manuales que les lastran en tiempo y fiabilidad. Por ejemplo, en el seguimiento de un expediente para mantener ciertas limitaciones o los cierres de caja y seguimiento de precios de los productos. Este proyecto toma como referencia aplicaciones de gestión de stock, facturación o erp's (Enterprice Resource Planning).
%\vspace{1em}
\par Los economatos sociales de manera similar a los bancos de alimentos, se apoyan fuertemente en el voluntariado de la gente, quienes en ocasiones son vecinos del mismo barrio en el que se encuentra. Dado que se dedican a ayudar a familias con pocos o ningún recurso. Es común encontrarlos en barrios marginales o con familias en peligro de exclusión social; esta es una de las razones por las que suelen carecer de equipos o procedimientos informatizados lo suficientemente desarrollados como para que el buen funcionamiento de éstos deje de consistir una carga y una preocupación.
%\vspace{1em}
\par Los economatos sociales suelen ser sedes de distribución de alimentos y productos de higiene, que se mantienen a base de donaciones que pueden provenir de particulares u Organizaciones No Gubernamentales (ONG) para mantener el stock de sus almacenes. El economato social para el que se ha realizado este TFG, está bajo la dirección de Cáritas y distribuye alimentos para la zona norte de Alicante. A priori, la aplicación dará solución exclusivamente a este economato, hasta que se haya refinado y adoptado lo suficiente como para dar el salto a otros economatos.
%\vspace{1em}
\par La entrada de soluciones de software desarrolladas adhoc para esta clase de actividades empuja en la buena dirección, ayudando a que la gente que se preocupa de poder atender a las familias lo mejor posible, puedan realizar su labor con mayor facilidad; sin tener que preocuparse de poner parches procedimentales a la forma en la que trabajan, que si bien ayudan, no solucionan. Normalmente hacen un uso principal del elemento humano, confiando en el buen hacer y organización personal de cada uno de los y las voluntarias. Algo de lo que se aprende en el grado de informática es que el elemento humano da pie inequívocamente a errores de procedimiento y datos; de esta forma, la automatización de recogida, procesamiento, almacenamiento y distribución de datos se hace cargo de una de las obligaciones manuales que estaban llevando a cabo en persona.
%\vspace{1em}
\par Del economato para el que hemos desarrollado este proyecto hemos visto algunos de estos procedimientos manuales, hojas de excel que se deben guardar apropiadamente y almacenar en carpetas concretas manualmente para su posterior uso, dado que una de las necesidades que tiene este economato es el seguimiento de expedientes. Pues cada familia tiene asignada ciertos límites en función de su situación personal, límites que establece la junta directiva del proyecto, como son saldo máximo a usar en el mes en curso y en relación a este, stock máximo de productos concretos que se puede retirar. Hacer uso de seguimientos manuales mientras, en el mismo momento en que se está atendiendo a esta familia, hay una cola interminable en la calle esperando; es cuanto menos estresante y, además, ineficiente y propenso a errores.
%\vspace{1em}
 
\section{Motivaciones}
Históricamente, España y su sociedad, ha mostrado interés en asuntos sociales, dedicando recursos económicos y personales en amparo de los más vulnerables. Aún existen actividades tan importantes como las que se llevan a cabo en los economatos sociales que están rozando la superficie del uso de la capacidad informática y tecnológica de la que disponemos actualmente. Esto empieza a hacer sentir urgente la necesidad de integraciones hechas a medida, que les brinden reducciones de tiempos en sus tareas, automatización de tareas repetitivas y la seguridad e instantaneidad de obtención y cálculo de datos veraces.
%\vspace{1em}
\par A día de hoy, la falta de recursos económicos no debería se excusa ni impedimento para tener soluciones de software de libre acceso, código abierto y de infraestructura en servicios en la nube. Una de las primeras premisas bajo las cuales se estableció la base de actuación y diseño de la aplicación para este trabajo, fue la falta de recursos económicos. No solo el dinero representa un problema, a menudo la burocracia hace tediosa la adopción de alguna solución informática, que por no ver claro su uso y beneficio, o la falta de conocimiento o capacidad de
aplicación y se acaba dejando de lado. De ésta manera, un proyecto libre y gratuito, saltaría de golpe un muro que aparentemente parecía duro de salvar.
%\vspace{1em}
\par Vista esta problemática, es un ingeniero de software la mejor carta a jugar para investigar y construir una solución que abarque estas premisas y así, facilitar y mejorar el trato y la calidad que reciben familias con
%pocos recursos
dificultades. Este proyecto tiene como objetivo agilizar y mejorar un proceso que puede ser duro para una familia en este tipo de situación, a quienes se les intenta dar una asistencia donde al menos no deban
%quienes lo último que quieren es poder comer en el mes en curso y no 
preocuparse de que se haya perdido un fichero o de que no encuentran el historial de su expediente.
Esto, a su vez, facilita el trabajo de los voluntarios, quienes pueden confiar en una herramienta que les brinda un proceso limpio y automático para realizar su función en el economato. Además, se verán aliviados aunque sea un poco, de la realidad que viven cada día que levantan la persiana del almacén, ya que podrán relajar el peso de la incertidumbre y la preocupación de hacer mal
su función como voluntarios.
%\vspace{1em}
\par Todo esto es lo que ha motivado a mis tutores a proponer éste como trabajo de fin de grado y a mí, como estudiante de informática, a solicitar llevarlo a cabo y defenderlo. Si bien he tenido la suerte de no necesitar nunca acudir a un banco de alimentos, sí he vivido de cerca la ansiedad y la carga que supone la falta de recursos. Ser voluntario puede ser, además de dar tu tiempo en trabajo manual, dar tiempo y conocimiento para crear una solución que les facilite el trabajo.

\section{Objetivos}
En este proyecto se pretende dar una solución de software mediante la consecución de una serie de objetivos propuestos, que permitan alcanzar la meta final de este trabajo:
\begin{itemize}
    \item Investigar la viabilidad de selfhosting con una raspberry pi
    \item Investigar la viabilidad de servicios en la nube y precios
    \item Comparar y decidir la arquitectura final que dará accesibilidad al servicio
    \item Analizar las necesidades del economato y decidir el alcance y funcionalidades
    \item Investigar capacidades y herramientas de leer códigos de barras (hardware, webcam)
    \item Diseño de una base de datos adecuada a las necesidades del economato
    \item Decidir tecnología para llevar a cabo la solución de software
    \item Implementar solución de software del backend
    \item Diseñar e implementar frontend SPA
    \item Reuniones con los implicados de forma ocasional para obtener feedback
    \item Entregar solución software cerrada y funcionando, infraestructura, backend y frontend
\end{itemize}

\section{Relación con asignaturas}
A los estudiantes de Ingeniería Informática en la Universidad de Alicante se los prepara para ser profesionales. Para conseguirlo, se les equipa con una sólida y amplia formación que los encamine a ser capaces de dirigir y realizar las tareas de que son propias de todos las fases del ciclo del software, arquitectura de hardware y gestión de proyectos. Los estudiantes, son preparados para resolver problemas de cualquier ámbito de la tecnología de la información y las comunicaciones, haciendo uso tanto de conocimiento científico, como métodos y técnicas propias de la ingeniería.
En el aspecto personal, me he decantado siempre por el software y la administración de sistemas, motivo de que haya cursado el grado de Ingeniería Informática.
%\vspace{1em}
\par Una recopilación de las asignaturas cursadas que ayudan a que este trabajo se pueda realizar en tiempo y forma, son:
\begin{itemize}
    \item Diseño de bases de datos
    \item Herramientas avanzadas para el desarrollo de software
    \item Diseño de sistemas de software
    \item Ingeniería web
    \item Programación
    \item Administración de Sistemas Operativos y Redes de computadores
    \item Gestión proyectos informáticos
\end{itemize}
