%%%%%%%%%%%%%%%%%%%%%%%%%%%%%%%%%%%%%%%%%%%%%%%%%%%%%%%%%%%%%%%%%%%%%%%%
% Plantilla TFG/TFM
% Escuela Politécnica Superior de la Universidad de Alicante
% Realizado por: Jose Manuel Requena Plens
% Contacto: info@jmrplens.com / Telegram:@jmrplens
%%%%%%%%%%%%%%%%%%%%%%%%%%%%%%%%%%%%%%%%%%%%%%%%%%%%%%%%%%%%%%%%%%%%%%%%

\chapter{Introducción}
En este apartado se va a presentar el trabajo, se introducirá la situación social actual que ha hecho de este trabajo una necesidad, qué lo ha motivado y, por último, se relacionará el alcance del trabajo y las asignaturas cursadas en el grado de Ingeniería Informática de la Universidad de Alicante.
\vspace{1em}
\par El objetivo de este proyecto consiste en aportar una solución informática a los economatos sociales, dado que usan procedimientos manuales que les lastran en tiempo y confianza. Como por ejemplo el seguimiento de un expediente para mantener ciertas limitaciones o los cierres de caja y seguimiento de precios de los productos. Este proyecto toma como referencia aplicaciones de gestión de stock, facturación o erp's (Enterprices Resource Planning).
\vspace{1em}
\par Los economatos sociales, como bancos de alimentos, se apoyan fuertemente en el voluntariado de la gente, habitualmente éstos son vecinos del mismo barrio en el que se encuentra. Dado que se dedican a ayudar a familias con pocos o ningún recurso, es común encontrarlos en zonas de clase media/baja e incluso en barrios con familias en peligro de exclusión social; esta es una de las razones por las que suelen carecer de equipos o procedimientos informatizados lo suficientemente desarrollados como para que el buen funcionamiento de éstos deje de consistir una carga y una preocupación.
\vspace{1em}
\par Los economatos sociales suelen ser sedes de distribución de alimentos y productos de higiene, que usan el dinero del que les provee alguna ONG para mantener el stock en sus almacenes. El economato social para el que se ha realizado este TFG está bajo la dirección de Caritas y, a priori, la aplicación dará solución exclusivamente a este economato, hasta que se haya refinado y adoptado lo suficiente como para dar el salto a otros economatos de la ciudad de Alicante.
\vspace{1em}
\par La entrada de soluciones de software desarrolladas adhoc para esta clase de actividades empuja en la buena dirección, ayudando a que la gente que se preocupa de poder atender a las familias lo mejor posible, realmente se dedique a ello; sin tener que preocuparse de poner parches procedimentales a la forma en la que trabajan, que si bien ayudan, no solucionan. Normalmente hacen un uso principal del elemento humano, confiando en el buen hacer y organización personal de cada uno de los y las voluntarias. Algo de lo que se aprende en el grado de informática es que el elemento humano da pie inequívocamente a errores de procedimiento y datos; de esta forma, la automatización de recogida, procesamiento, almacenamiento y distribución de datos se hace cargo de una de las obligaciones manuales que estaban llevando a cabo en persona.
\vspace{1em}
\par Del economato para el que hemos trabajado este proyecto hemos visto algunos de estos procedimientos manuales, hojas de excel que se deben guardar apropiadamente y almacenar en carpetas concretas manualmente para su posterior uso, dado que una de las necesidades que tiene este economato es el seguimiento de expedientes. Pues cada familia tiene ciertas limitaciones en función de su situación social, limitaciones que decide Caritas, como son saldo máximo a usar en el mes en curso y en relación a este, stock máximo de productos concretos que se puede retirar. Hacer uso de seguimientos manuales mientas, en el mismo momento en que se está atendiendo a esta familia, hay una cola interminable en la calle esperando; es cuanto menos estresante y, además, ineficiente y propenso a errores.
\vspace{1em}
 
\section{Motivaciones}
Este país y nuestra sociedad, históricamente, ha sido social, dedicando recursos económicos y personales en amparo de los más vulnerables. Que aún haya actividades tan importantes, como las que se llevan a cabo en los economatos sociales, que estén rozando la superficie del uso de la capacidad informática y tecnológica de la que disponemos actualmente; empieza a hacer sentir urgente la necesidad de integraciones hechas a medida, que les brinden reducciones de tiempos en sus tareas, automatización de tareas repetitivas y la seguridad e instantaneidad de obtención y cálculo de datos veraces.
\vspace{1em}
\par A día de hoy, la falta de recursos económicos no es excusa ni debería ser impedimento para tener soluciones de software de libre acceso, código abierto y de infraestructura en servicios en la nube. Una de las primeras premisas, bajo las cuales, se estableció la base de actuación y diseño de la aplicación para este trabajo fue ésta: la falta de recursos económicos. No siempre por no tener dinero, si no que a menudo, la burocracia hace tan tediosa la adopción de alguna solución informática, que por no ver claro su uso y beneficio, o la falta de conocimiento o capacidad de
aplicación, se acaba dejando de lado; de ésta manera, un proyecto libre y gratuito, saltaría de golpe un muro que aparentemente se había convertido en uno francamente duro de salvar.
\vspace{1em}
\par Vista esta problemática, es un ingeniero de software la mejor carta a jugar para investigar y construir una solución que abarque estas premisas y así, facilitar y mejorar el trato y la calidad que reciben estas familias sin recursos; que lo único que quieren es poder comer en el mes en curso y no preocuparse de que se haya perdido un fichero o de que no encuentran el historial de su expediente y que les hagan esperar mientras llaman por teléfono a la sede para informarse.
Esto, a su vez, facilita el trabajo de los voluntarios; que pudiendo confiar en una herramienta que les va a brindar un proceso limpio y automático para realizar su función en el economato, se verán aliviados, aunque sea un poco, de la realidad que viven cada día que levantan la persiana del almacén; pues se habrán quitado de encima el peso de la incertidumbre y la preocupación de hacer mal
su función como voluntarios.
\vspace{1em}
\par Todo esto es lo ha motivado a mis tutores a proponer éste como trabajo de fin de grado y a mí, como estudiante de informática, a solicitar llevarlo a cabo y defenderlo. Si bien he tenido la suerte de no necesitar nunca acudir a un banco de alimentos, sí he vivido de cerca la ansiedad y la carga que supone la falta de recursos. Ser voluntario puede ser, además de dar tu tiempo en trabajo manual, dar tiempo y conocimiento para crear una solución que les facilite el trabajo.

\section{Objetivos}
En este proyecto se pretende dar una solución de software mediante la consecución de una serie de objetivos propuestos, que permitan alcanzar la meta final de este trabajo:
\begin{itemize}
    \item Investigar la viabilidad de selfhosting con una raspberry pi
    \item Investigar la viabilidad de servicios en la nube y precios
    \item Comparar y decidir la arquitectura final que dará accesibilidad al servicio
    \item Analizar las necesidades del economato y decidir el alcance y funcionalidades
    \item Investigar capacidades y herramientas de leer códigos de barras (hardware, webcam)
    \item Diseño de una base de datos adecuada a las necesidades del economato
    \item Decidir tecnología para llevar a cabo la solución de software
    \item Implementar solución de software del backend
    \item Diseñar e implementar frontend SPA
    \item Reuniones con los implicados de forma ocasional para obtener feedback
    \item Entregar solución software cerrada y funcionando, infraestructura, backend y frontend
\end{itemize}

\section{Relación con asignaturas}
Los estudiantes de la Ingeniería Informática de la Universidad de Alicante son preparados como profesionales, haciendo uso de una sólida y amplia formación que los encamine a ser capaces de dirigir y realizar las tareas de que son propias de todos las fases del ciclo del software, arquitectura de hardware y gestión de proyectos. Los estudiantes se preparan para resolver problemas de cualquier ámbito de la tecnología de la información y las comunicaciones, aplicando conocimiento científico y métodos y técnicas propios de la ingeniería.
En el aspecto personal, me he decantado siempre por el software y la administración de sistemas; motivo de que haya cursado el grado de Ingeniería Informática.
\vspace{1em}
\par Una recopilación de las asignaturas cursadas que ayudan a que este trabajo se pueda realizar en tiempo y forma, son:
\begin{itemize}
    \item Diseño de bases de datos
    \item Herramientas avanzadas para el desarrollo de software
    \item Diseño de sistemas de software
    \item Ingeniería web
    \item Programación
    \item Administración de Sistemas Operativos y Redes de computadores
    \item Gestión proyectos informáticos
\end{itemize}

\section{Borrar al terminar}
Dejo el contenido siguiente en el bloque 1.x como guía hasta que termine la redacción del TFG.

\section{Estructura de un \glsentryshort{tfg}}
\begin{description}
\item[Preámbulo:] se describirán brevemente la motivación que ha originado la realización del \gls{tfg}/\gls{tfm}, así como una breve descripción de los objetivos generales que se quieren alcanzar con el trabajo presentado.
\item[Agradecimientos:] se podrán añadir las hojas necesarias para realizar los agradecimientos, a veces obligatorios, a las entidades y organismos colaboradores.
\item[Dedicatoria:] se podrá añadir una única hoja con dedicatorias, su alineación será derecha.
\item[Citas:] (frases célebres) se podrá añadir una única hoja con citas, su alineación será derecha.
\item[Índices:] cada índice debe comenzar en una nueva página, se incluirán los índices que se estimen necesarios (conforme UNE 50111:1989), en este orden:
\begin{description}
\item[Índice de contenidos:] (obligatorio siempre) se incluirá un índice de las secciones de las que se componga el documento, la numeración de las 
divisiones y subdivisiones utilizarán cifras arábigas (según UNE 50132:1994) y harán mención a la página del documento donde se ubiquen.
\item[Índice de figuras:] si el documento incluye figuras se podrá incluir también un índice con su relación, indicando la página donde se ubiquen.
\item[Índice de tablas:] en caso de existir en el texto, ídem que el anterior.
\item[Índice de abreviaturas, siglas, símbolos, etc.:] en caso de ser necesarios se podrán incluir cada uno de ellos.
\end{description}
\item[Cuerpo del documento:] en el contenido del documento se da flexibilidad para su organización y se puede estructurar en las secciones que se considere. En todo caso obligatoriamente se deberá, al menos, incluir los siguientes contenidos:
\begin{description}
\item[Introducción:] donde se hará énfasis a la importancia de la temática, su vigencia y actualidad; se planteará el problema a investigar, así como el propósito o finalidad de la investigación.
\item[Marco teórico o Estado del arte:] se hará mención a los elementos conceptuales que sirven de base para la investigación, estudios previos relacionados con el problema planteado, etc.
\item[Objetivos:] se establecerán el objetivo general y los específicos.
\item[Metodología:] se indicarán el tipo o tipos de investigación, las técnicas y los procedimientos que serán utilizados para llevarla a cabo; se identificarán la población y el tamaño de la muestra así como las técnicas e instrumentos de recolección de datos.
\item[Resultados:] incluirá los resultados de la investigación o trabajo, así como el análisis y la discusión de los mismos.
\end{description}
\item[Conclusiones:] obligatoriamente se incluirá una sección de conclusiones donde se realizará un resumen de los objetivos conseguidos así como de los resultados obtenidos si proceden.
\item[Bibliografía y referencias:] se incluirá también la relación de obras y materiales consultados y empleados en la elaboración de la memoria del \gls{tfg}/\gls{tfm}. La bibliografía y las referencias serán indexadas en orden alfabético (sistema nombre y fecha) o se numerará correlativamente según aparezca (sistema numérico). Se empleará la familia 1 como tipo de letra. Podrá utilizarse cualquier sistema bibliográfico normalizado predominante en la rama de conocimiento, estableciéndose como prioritarios el sistema ISO 690, sistema \gls{apa}  o Harvard (no necesariamente en ese orden de preferencia). En esta plantilla Latex se propone usar el estilo \gls{apa} indicándolo en la línea correspondiente como 
\begin{verbatim}
\bibliographystyle{apacite}
\end{verbatim}
\item[Anexos:] se podrán incluir los anexos que se consideren oportunos.
\end{description}

\section{Citar bibliografía}
Para citar la bibliografía tal como se define en el sistema APA (en esta web se indica como debe aparecer en el texto la cita: \url{http://guides.libraries.psu.edu/apaquickguide/intext}) se debe realizar con alguno de los comandos mostrados a continuación:

\begin{lstlisting}[style=Latex-color]
Esto es una cita estándar: \citet{Shaw1996}, que también puedes mostrar con paréntesis así: \citep{Shaw1996}. También se puede realizar una cita indicando a qué parte te refieres \citep[ver][Cap. 2]{Shaw1996} o \citep[Cap. 2]{Shaw1996} o \citep[ver][]{Shaw1996}. 

También puedes mostrar todos los autores cuando hay más de 2 autores añadiendo un asterisco después del comando como: \citet*{Akyildiz2005}, sin el asterisco quedaría así: \citet{Akyildiz2005}.

O puedes citar dos o más fuentes al mismo tiempo: \citep{Barkan1995,Leighton2012}

\end{lstlisting}
Y \LaTeX~genera lo siguiente:
\\
\par Esto es una cita estándar: \citet{Shaw1996}, que también puedes mostrar con paréntesis así: \citep{Shaw1996}. También se puede realizar una cita indicando a qué parte te refieres \citep[ver][Cap. 2]{Shaw1996} o \citep[Cap. 2]{Shaw1996} o \citep[ver][]{Shaw1996}. 
\\
\par También puedes mostrar todos los autores cuando hay más de 2 autores añadiendo un asterisco después del comando como: \citet*{Akyildiz2005}, sin el asterisco quedaría así: \citet{Akyildiz2005}.
\\
\par O puedes citar dos o más fuentes al mismo tiempo: \citep{Barkan1995,Leighton2012}


\section{Notas a pie de página}

Para introducir notas a pie de página se debe escribir lo siguiente:

\begin{lstlisting}[style=Latex-color]
	La plantilla necesita el motor XeLaTeX \footnote{Para más información sobre XeLaTeX visita \url{https://es.sharelatex.com/learn/XeLaTeX}} (el más recomendable actualmente), por lo que si el programa que utilizas compila la plantilla con el motor pdfLaTeX \footnote{También puedes buscar más información en internet} (el más habitual pero menos potente) debes cambiarlo por XeLaTeX en las opciones del programa. Si no sabes como hacerlo busca en el manual del programa o en google.
\end{lstlisting}

\LaTeX~genera lo siguiente (observa las notas a pie de página):
\\
\par La plantilla necesita el motor XeLaTeX\footnote{Para más información sobre XeLaTeX visita \url{https://es.sharelatex.com/learn/XeLaTeX}} (el más recomendable actualmente), por lo que si el programa que utilizas compila la plantilla con el motor pdfLaTeX\footnote{También puedes buscar más información en internet} (el más habitual pero menos potente) debes cambiarlo por XeLaTeX en las opciones del programa. Si no sabes como hacerlo busca en el manual del programa o en google.
\section{Estilos de texto}

A continuación se muestran ejemplos de distintos estilos de texto:

\begin{itemize}
	\item \textbackslash textit\{Cursiva\} $\rightarrow$ \textit{Cursiva}
	\item \textbackslash emph\{Cursiva 2\} $\rightarrow$ \emph{Cursiva 2}
	\item \textbackslash textbf\{Negrita\} $\rightarrow$ \textbf{Negrita}
	\item \textbackslash texttt\{Monoespacio\} $\rightarrow$ \texttt{Monoespacio}
	\item \textbackslash textsc\{Mayúsculas capitales\} $\rightarrow$ \textsc{Mayúsculas capitales}
	\item \textbackslash uppercase\{Todo mayúsculas\} $\rightarrow$ \uppercase{Todo mayúsculas} 
\end{itemize}
