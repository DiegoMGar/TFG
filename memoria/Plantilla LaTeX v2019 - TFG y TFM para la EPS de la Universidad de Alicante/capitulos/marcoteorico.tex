%%%%%%%%%%%%%%%%%%%%%%%%%%%%%%%%%%%%%%%%%%%%%%%%%%%%%%%%%%%%%%%%%%%%%%%%
% Plantilla TFG/TFM
% Escuela Politécnica Superior de la Universidad de Alicante
% Realizado por: Jose Manuel Requena Plens
% Contacto: info@jmrplens.com / Telegram:@jmrplens
%%%%%%%%%%%%%%%%%%%%%%%%%%%%%%%%%%%%%%%%%%%%%%%%%%%%%%%%%%%%%%%%%%%%%%%%

\chapter{Estado de arte}
En este apartado se va a presentar el impacto y el beneficio de la tecnología entrando, cada vez más, en el campo práctico de los servicios sociales.
\vspace{1em}
\par La tecnología ha revolucionado nuestra forma de consumir, de relacionarnos y de informarnos. Donde tampoco han sido ajenos a la evolución de la tecnología, es en el ámbito de los servicios sociales. Sin embargo, comparándolo con otros sectores mucho más maduros y con más presupuesto como la banca o el comercio, el sector social parece no haber sido capaz de adoptar o tener acceso a toda la tecnología que ya está desarrollada, afectando negativamente al servicio humano que se pretende dar.
\vspace{1em}
\par La tecnología en los servicios sociales tiene un papel cada vez mayor, éste puede adoptar muchas formas, como el uso de inteligencia artificial, los sistemas de gestión de casos, servicios especialmente desarrollados para hacer uso de la tecnología de asistencia. Como se estaba explicando, éstos avances tecnológicos pueden ayudar a mejorar la planificación, gestión y prestación de servicios sociales, pero también es importante comprender los desafíos que plantea la digitalización, como la falta de conocimiento sobre las nuevas tecnologías, su costo y cómo garantizar la protección de la privacidad y la seguridad.
\vspace{1em}
\par Aquí entraría en juego la Universidad de Alicante con su plan formativo de Ingeniería Informática, formando personal con los conocimientos y habilidades necesarias para ayudar acotar, acortar y amortiguar los miedos que suponen en este sector la aproximación a la tecnología.

\section{La importancia de la tecnología}
Pese a que la crisis del COVID-19 pueda resultarnos nueva, sin embargo las crisis medioambientales y de salud han sido una constante en nuestro planeta. Éstas siempre han superado las agendas planificadas y los objetivos establecidos de las organizaciones comunitarias para continuar ejerciendo labores de manera tradicional y más importante aún, para coordinar asistencia social y llegar a los más necesitados. Por ejemplo en Puerto Rico se han experimentado fenómenos que han derrotado cualquier intento de gestión, desde el Huracán María hasta el terremoto de 2020, resultando en una falta de atención a las necesidades sociales.
\vspace{1em}
\par Consecuencia de este tipo de crisis, las organizaciones de impacto comunitario se ven obligadas a redirigir sus esfuerzos hacia nuevas maneras de alcanzar sus objetivos. Esta problemática, a su vez, se extiende a los individuos que más lo necesitan, dado que encuentran dificultades añadidas para encontrar ayuda en estas circunstancias.
\clearpage
Esta realidad no sólo atañe a las organizaciones sin fines de lucro, nos debería afectar a todos como sociedad, a las corporaciones, entidades privadas y gubernamentales.
\vspace{1em}
\par Los servicios sociales son la herramienta principal para una integración apropiada de las comunicaciones desfavorecidas y de los individuos en desventaja, resultando en un fortalecimiento del tejido social. Frente a este tipo de situaciones los métodos tradicionales parecen no ser suficientes, las organizaciones comunitarias necesitan poder medir datos de forma veraz y con precisión, asegurar la continuidad de sus servicios, atender de forma rápida y eficaz, dedicar más esfuerzo a las personas que a los procesos.
\vspace{1em}
\par Es por eso que en una sociedad cada vez más activa en el mundo digital, es necesario adquirir nuevas herramientas que sean contemporáneas y relevantes para tener resultados eficientes. La tecnología permite anticiparse al impacto de situaciones futuras, las organizaciones sociales pueden tener información suficiente para no necesitar realizar un estudio de necesidad para cada desastre que pueda suceder, por el contrario, pueden usar la tecnología para tener identificadas las necesidades previamente y, así, atendiendo de antemano los casos y agilizando la ayuda social.
\vspace{1em}
\par Para beneficiarse de los aspectos digitales mencionados, se debe integrar la tecnología en los procesos de coordinación social. Ejemplos podrían ser:
\begin{enumerate}
    \item El uso de la tecnología para agilizar la asistencia social
    \item La conexión entre el individuo que necesita ayuda y la organización que puede asistirle
    \item Conexión entre organizaciones aliadas, de interés o de apoyo
    \item Mapas interactivos para ubicar a las organizaciones por región y pueblo
    \item Manejo de casos y administración de recursos
\end{enumerate}